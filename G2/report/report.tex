\documentclass[11pt]{article}
\usepackage[a4paper, hmargin={2.8cm, 2.8cm}, vmargin={2.5cm, 2.5cm}]{geometry}
\usepackage{eso-pic} % \AddToShipoutPicture
\usepackage{graphicx} % \includegraphics
\usepackage{fancyhdr, amsmath, amssymb, comment, caption, placeins, subfigure,
    fixltx2e, changepage, listings, courier, soul, hyperref, geometry,
    enumerate, listings, enumitem}
\usepackage[T1]{fontenc}
\usepackage[utf8]{inputenc}
\usepackage[english]{babel}


\author{\Large{Magnus N\o rskov Stavngaard} \\
		\texttt{magnus@stavngaard.dk}
		\\\\
		\Large{Mark Jan Jacobi} \\
        \texttt{mark@jacobi.pm}
		 \\\\
		\Large{Mads Ynddal} \\
		\texttt{mynddal@me.com}
}

\lstdefinestyle{customc}{
  belowcaptionskip=1\baselineskip,
  breaklines=true,
  xleftmargin=\parindent,
  language=C,
  showstringspaces=false,
  keywordstyle=\bfseries\color{green!40!black},
  morekeywords={size_t,node,heap},
  commentstyle=\itshape\color{purple!40!black},
  identifierstyle=\color{blue},
  stringstyle=\color{orange},
}

\lstset{
basicstyle=\footnotesize,
language=C,
numbers=none}
\lstset{literate=
  {á}{{\'a}}1 {é}{{\'e}}1 {í}{{\'i}}1 {ó}{{\'o}}1 {ú}{{\'u}}1
  {Á}{{\'A}}1 {É}{{\'E}}1 {Í}{{\'I}}1 {Ó}{{\'O}}1 {Ú}{{\'U}}1
  {à}{{\`a}}1 {è}{{\`e}}1 {ì}{{\`i}}1 {ò}{{\`o}}1 {ù}{{\`u}}1
  {À}{{\`A}}1 {È}{{\'E}}1 {Ì}{{\`I}}1 {Ò}{{\`O}}1 {Ù}{{\`U}}1
  {ä}{{\"a}}1 {ë}{{\"e}}1 {ï}{{\"i}}1 {ö}{{\"o}}1 {ü}{{\"u}}1
  {Ä}{{\"A}}1 {Ë}{{\"E}}1 {Ï}{{\"I}}1 {Ö}{{\"O}}1 {Ü}{{\"U}}1
  {â}{{\^a}}1 {ê}{{\^e}}1 {î}{{\^i}}1 {ô}{{\^o}}1 {û}{{\^u}}1
  {Â}{{\^A}}1 {Ê}{{\^E}}1 {Î}{{\^I}}1 {Ô}{{\^O}}1 {Û}{{\^U}}1
  {œ}{{\oe}}1 {Œ}{{\OE}}1 {æ}{{\ae}}1 {Æ}{{\AE}}1 {ß}{{\ss}}1
  {ç}{{\c c}}1 {Ç}{{\c C}}1 {ø}{{\o}}1 {å}{{\r a}}1 {Å}{{\r A}}1
  {€}{{\EUR}}1 {£}{{\pounds}}1
}

\title{
    \vspace{3cm}
    \Huge{OSM} \\
    \Large{G1}
}

\pagestyle{fancy}
\lhead{\small{Magnus S. Mark J. Mads Y.}}
\chead{\date{\today}}
\rhead{University of Copenhagen}
% \lfoot{}
% \cfoot{}
% \rfoot{}

% Change indent length of paragraph not after a header.
\setlength{\parindent}{0cm}

% Remove page numbering in the beginning
\pagenumbering{gobble}

\begin{document}
    %% Change `ku-farve` to `nat-farve` to use SCIENCE's old colors or
    %% `natbio-farve` to use SCIENCE's new colors and logo.
    \AddToShipoutPicture*{\put(0,0){\includegraphics*[viewport=0 0 700 600]
        {include/ku-farve}}}
    \AddToShipoutPicture*{\put(0,602){\includegraphics*[viewport=0 600 700 1600]
        {include/ku-farve}}}

    %% Change `ku-en` to `nat-en` to use the `Faculty of Science` header
    \AddToShipoutPicture*{\put(0,0){\includegraphics*{include/ku-en}}}

    \clearpage
    \maketitle
    \thispagestyle{empty}

    \newpage

    %\tableofcontents

    %\newpage

    \pagenumbering{arabic} % Arabic page numbers (and reset to 1)

\section*{Task 1 - A Priority Queue}
We were asked to implement a heap priority queue in the programming language C.
The heap should handle any kind of data and should hold as many elements as the
user wants.  This means that the heap should expand when needed and use void
pointers to save elements on the heap.

\subsection*{Function Description and Assumptions}
The heap is represented with this structure,

\begin{lstlisting}[style=customc]
typedef struct {
  node *root;
  size_t size;
  size_t alloc_size;
} heap;
\end{lstlisting}

\textit{Description} \\
The heap contains an array of node pointers, the size of the heap in number of
elements, and the size of the allocated space in bytes. \\
\textit{Assumptions} \\
We assumed that root is an array. \\

A node in the array in the heap is represented by the struct,

\begin{lstlisting}[style=customc]
typedef struct {
  void *thing;
  int priority;
} node;
\end{lstlisting}

\textit{Description}\\
The void pointer "thing" is a pointer to the element queued, and the integer
priority contains the priority of a given node.  The higher the value, the
higher priority. \\
\textit{Assumptions}\\
We assumed we were building a max-heap which is why a greater numerical value of
priority is assumed to correspond to a higher priority. \\

\begin{lstlisting}[style=customc]
void heap_initialize(heap *);
\end{lstlisting}
\textit{Description} \\
This function initializes the heap, malloc is called to allocate space for the
nodes and the length of the heap and the allocated length is set.

\begin{lstlisting}[style=customc]
void heap_clear(heap *);
\end{lstlisting}
\textit{Description}\\
Frees the space used to store the nodes in the heap, so when the struct goes out
of scope, no memory is associated with the heap anymore.

\begin{lstlisting}[style=customc]
size_t heap_size(heap *);
\end{lstlisting}

\textit{Description}\\
Returns the amount of elements in the heap. (stored under size in the heap
structure). \\

\begin{lstlisting}[style=customc]
void * heap_top(heap *);
\end{lstlisting}

\textit{Description} \\
Returns the pointer for the thing in the highest priority node.
\textit{Assumptions}\\
Since this function had a void pointer return type instead of a node pointer, we
assumed that only the void pointer "thing" was wanted for return, and not the
whole node element.  We simply return the first element in the array since the
heap property dictates that that is where the highest priority element is. \\

\begin{lstlisting}[style=customc]
void heap_insert(heap *, void *, int);
\end{lstlisting}
\textit{Description}\\
Inserts an element into the heap.  The function arguments is a heap pointer, a
void pointer for the thing in the node type, and an integer for the priority. \\

\begin{lstlisting}[style=customc]
void * heap_pop(heap *);
\end{lstlisting}
\textit{Description}\\
Returns the a void pointer to the element in the heap with the highest priority.
The returned element is removed from the heap, and the heap is updated so the
next highest priority element is now on top.
\textit{Assumptions}\\
Again, we assume that the returned pointer is not the node element, but the
pointer for the thing in the node type. \\

\subsection*{Running time and algorithm}
For most of the functions, the running time is constant, in these functions, we
do not iterate over the elements, or in some way make the code independent on
the size of the heap. therefore it is constant. We also assume that malloc space
allocation is not time dependent of the size allocated. \\
The functions of interest is heap\_insert and heap\_pop. these have $O(log n)$
running times.

\subsubsection*{heap\_insert}
heap\_insert uses to bottom to top heapify. When an element is inserted, it will
be at the very end of the node array.  From there, it will be compared to its
parent node, if the parent node have a lower priority, the algorithm will flip
their position in the heap, and run the same process again, with focus on the
same node, but at its new position.  The algorithm stops if the parent node is
bigger, or if the node becomes the top node.

\subsubsection*{heap\_pop}
heap\_pop uses top to bottom heapify.  First the top element is returned, then
it is flipped with the end element in the array, and the size of the array is
lowered by one. We could just copy the last element to the first one instead of
flipping them, but for simplicity we flip them like the insert algorithm. \\
When the flip has occurred, we focus on the root node of the array.  We then
look at the nodes children.  If They exist, we check if they are bigger.  If
both are bigger, we flip with the biggest one.  This algorithm runs until the
nodes underneath is smaller, or non existent.

\section*{Task 2 - syscall read/write}
In this assignment we were asked to implement read and write to standard in and
out for the Buenos operating system. \\

Our implementation of \textit{syscall\_read} and \textit{syscall\_write} was
written inspired by the code in \textit{init\_startup\_fallback} from
\textit{init/main.c}.  The implementation should be easily expandable to be able
to read and write using the filehandle argument given to the read and write
functions.

In \textit{syscall.h} the syscall numbers are already defined for the read and
write function as,

\begin{lstlisting}[style=customc]
#define SYSCALL\_READ  0x204
#define SYSCALL\_WRITE 0x205
\end{lstlisting}

Therefore we created a case for these system calls in the file
\textit{syscall.c}.

\begin{lstlisting}[style=customc]
switch (user_context->cpu_regs[MIPS_REGISTER_A0]) {
    case SYSCALL_HALT:
        halt_kernel();
        break;
    case SYSCALL_READ:
        user_context->cpu_regs[MIPS_REGISTER_V0] =
        syscall_read(
            (int) user_context->cpu_regs[MIPS_REGISTER_A1],
            (void *) user_context->cpu_regs[MIPS_REGISTER_A2],
            (int) user_context->cpu_regs[MIPS_REGISTER_A3]
        );

        break;
    case SYSCALL_WRITE:
        user_context->cpu_regs[MIPS_REGISTER_V0] =
        syscall_write(
            (int) user_context->cpu_regs[MIPS_REGISTER_A1],
            (void const *) user_context->cpu_regs[MIPS_REGISTER_A2],
            (int) user_context->cpu_regs[MIPS_REGISTER_A3]
        );

        break;
    default:
        KERNEL_PANIC("Unhandled system call\n");
}
\end{lstlisting}

The syscall number is saved in the MIPS register A0, therefore a switch is
performed on these system calls so we handle the correct system call.  The
arguments to the read and write functions can be found in the MIPS registers
A1, A2 and A3, and the return value of the function should be saved in the MIPS
register V0.  The only thing done in the \textit{syscall.c} file is casting the
arguments to the correct types and saving the result of the function call the V0
register.

In \textit{IO.c} we initialize read by accepting only reading from standard in
(for simplicity) if a user try and read from anywhere else $-1$ is returned to
signify an error.  After this we create a device handler for TTY which we can
use to get a GCD, Generic Character Device which we use to read and write from a
TTY.  For this we simply use the already defined \emph{gcd->read()} and
\emph{gcd->write()} functions.  Below this we show our implementation of read,
the write implementation is almost the same.

\begin{lstlisting}[style=customc]
int syscall_read(int fhandle, void *buffer, int length)
{
    device_t *dev;
    gcd_t *gcd;

    // Vi vil kun bruge STDIN på nuværende tidspunkt
    if (fhandle != FILEHANDLE_STDIN)
        return -1;

    // Opret en device handle til TTY
    dev = device_get(YAMS_TYPECODE_TTY, 0);
    KERNEL_ASSERT(dev != NULL);

    // Opret generic character device ud fra dev
    gcd = (gcd_t *) dev->generic_device;
    KERNEL_ASSERT(gcd != NULL);

    // Skriv til TTY med read metoden
    return gcd->read(gcd, buffer, length);
}
\end{lstlisting}

We were also asked to test our implementation, we do that with a short test file
in the tests folder called \textit{readwrite.c}.

\begin{lstlisting}[style=customc]
#include "tests/lib.h"
#include "proc/syscall.h"

int main(void)
{
    char buffer[128];

    syscall_write(FILEHANDLE_STDOUT, "Hi, press a key!\n", 18);
    syscall_read(FILEHANDLE_STDIN, buffer, 127);
    syscall_write(FILEHANDLE_STDOUT, "Pressed button was, ", 21);
    syscall_write(FILEHANDLE_STDOUT, buffer, 128);
    syscall_write(FILEHANDLE_STDOUT, "\n", 1);

    syscall_halt();

    return 0;
}
\end{lstlisting}

\end{document}
