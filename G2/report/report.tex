\documentclass[11pt]{article}
\usepackage[a4paper, hmargin={2.8cm, 2.8cm}, vmargin={2.5cm, 2.5cm}]{geometry}
\usepackage{eso-pic} % \AddToShipoutPicture
\usepackage{graphicx} % \includegraphics
\usepackage{fancyhdr, amsmath, amssymb, comment, caption, placeins, subfigure,
    fixltx2e, changepage, listings, courier, soul, hyperref, geometry,
    enumerate, listings, enumitem}
\usepackage[T1]{fontenc}
\usepackage[utf8]{inputenc}
\usepackage[english]{babel}


\author{\Large{Magnus N\o rskov Stavngaard} \\
		\texttt{magnus@stavngaard.dk}
		\\\\
		\Large{Mark Jan Jacobi} \\
        \texttt{mark@jacobi.pm}
		 \\\\
		\Large{Mads Ynddal} \\
		\texttt{mynddal@me.com}
}

\lstdefinestyle{customc} {
    belowcaptionskip=1\baselineskip,
    breaklines=true,
    xleftmargin=\parindent,
    language=C,
    showstringspaces=false,
    keywordstyle=\bfseries\color{green!40!black},
    morekeywords={size_t,node,heap},
    commentstyle=\itshape\color{purple!40!black},
    identifierstyle=\color{blue},
    stringstyle=\color{orange},
}

\lstset{
basicstyle=\footnotesize,
language=C,
numbers=none}
\lstset{literate=
    {á}{{\'a}}1 {é}{{\'e}}1 {í}{{\'i}}1 {ó}{{\'o}}1 {ú}{{\'u}}1
    {Á}{{\'A}}1 {É}{{\'E}}1 {Í}{{\'I}}1 {Ó}{{\'O}}1 {Ú}{{\'U}}1
    {à}{{\`a}}1 {è}{{\`e}}1 {ì}{{\`i}}1 {ò}{{\`o}}1 {ù}{{\`u}}1
    {À}{{\`A}}1 {È}{{\'E}}1 {Ì}{{\`I}}1 {Ò}{{\`O}}1 {Ù}{{\`U}}1
    {ä}{{\"a}}1 {ë}{{\"e}}1 {ï}{{\"i}}1 {ö}{{\"o}}1 {ü}{{\"u}}1
    {Ä}{{\"A}}1 {Ë}{{\"E}}1 {Ï}{{\"I}}1 {Ö}{{\"O}}1 {Ü}{{\"U}}1
    {â}{{\^a}}1 {ê}{{\^e}}1 {î}{{\^i}}1 {ô}{{\^o}}1 {û}{{\^u}}1
    {Â}{{\^A}}1 {Ê}{{\^E}}1 {Î}{{\^I}}1 {Ô}{{\^O}}1 {Û}{{\^U}}1
    {œ}{{\oe}}1 {Œ}{{\OE}}1 {æ}{{\ae}}1 {Æ}{{\AE}}1 {ß}{{\ss}}1
    {ç}{{\c c}}1 {Ç}{{\c C}}1 {ø}{{\o}}1 {å}{{\r a}}1 {Å}{{\r A}}1
    {€}{{\EUR}}1 {£}{{\pounds}}1
}

\title{
    \vspace{3cm}
    \Huge{OSM} \\
    \Large{G1}
}

\pagestyle{fancy}
\lhead{\small{Magnus S. Mark J. Mads Y.}}
\chead{\date{\today}}
\rhead{University of Copenhagen}
% \lfoot{}
% \cfoot{}
% \rfoot{}

% Change indent length of paragraph not after a header.
\setlength{\parindent}{0cm}

% Remove page numbering in the beginning
\pagenumbering{gobble}

\begin{document}
    %% Change `ku-farve` to `nat-farve` to use SCIENCE's old colors or
    %% `natbio-farve` to use SCIENCE's new colors and logo.
    \AddToShipoutPicture*{\put(0,0){\includegraphics*[viewport=0 0 700 600]
        {include/ku-farve}}}
    \AddToShipoutPicture*{\put(0,602){\includegraphics*[viewport=0 600 700 1600]
        {include/ku-farve}}}

    %% Change `ku-en` to `nat-en` to use the `Faculty of Science` header
    \AddToShipoutPicture*{\put(0,0){\includegraphics*{include/ku-en}}}

    \clearpage
    \maketitle
    \thispagestyle{empty}

    \newpage

    %\tableofcontents

    %\newpage

    \pagenumbering{arabic} % Arabic page numbers (and reset to 1)

    \section{Types and functions for userland processes in Buenos}
    In this assignment we were asked to implement userland processes and the
    library the kernel use when managing the userland processes.  We first
    implemented a datastructure in \textit{proc/process.h} representing a
    userland process.

    \begin{lstlisting}[style=customc]
typedef struct {
    process_id_t pid;                        /* ID of this process. */
    char executable[PROCESS_MAX_FILELENGTH]; /* The name of the file run. */
    int exit_code;                           /* The return value. */
    process_state_t process_state;           /* The state of the process. */
    process_id_t parent;                     /* ID of the parent process. */
} process_control_block_t;
    \end{lstlisting}

    A process contains a name of an executable file that it executes.  It
    contains a unique id used to identify it in a table of all processes.  It
    contains an exit code which is set when the process exits. It contains a
    process state, which we define by an enumerator, examples of values is
    \texttt{PROCESS_ZOMBIE}, \texttt{PROCESS_DEAD}, \texttt{PROCESS_RUNNING}.
    Lastly a process contains a pid which references its parent process. \\

    In the file \textit{process.c} we implemented the function managing the
    process table.  When a user creates a process,

\end{document}
