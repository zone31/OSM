\documentclass[11pt]{article}
\usepackage[a4paper, hmargin={2.8cm, 2.8cm}, vmargin={2.5cm, 2.5cm}]{geometry}
\usepackage{eso-pic} % \AddToShipoutPicture
\usepackage{graphicx} % \includegraphics
\usepackage{fancyhdr, amsmath, amssymb, comment, caption, placeins, subfigure,
    fixltx2e, changepage, listings, courier, soul, hyperref, geometry,
    enumerate, listings, enumitem}


\author{\Large{Magnus N\o rskov Stavngaard} \\
		\texttt{magnus@stavngaard.dk}
		\\\\
		\Large{Mark Jan Jacobi} \\
        \texttt{mark@jacobi.pm}
		 \\\\
		\Large{Mads Ynddal} \\
		\texttt{template@russianwomanlookingforyou.ru}
}

\lstdefinestyle{customc}{
  belowcaptionskip=1\baselineskip,
  breaklines=true,
  xleftmargin=\parindent,
  language=C,
  showstringspaces=false,
  keywordstyle=\bfseries\color{green!40!black},
  morekeywords={size_t,node,heap},
  commentstyle=\itshape\color{purple!40!black},
  identifierstyle=\color{blue},
  stringstyle=\color{orange},
}

\lstset{
basicstyle=\footnotesize,
language=C,
numbers=none}

\title{
    \vspace{3cm}
    \Huge{OSM} \\
    \Large{G1}
}

\pagestyle{fancy}
\lhead{\small{Magnus S. Mark J. Mads Y.}}
\chead{\date{\today}}
\rhead{University of Copenhagen}
% \lfoot{}
% \cfoot{}
% \rfoot{}

% Change indent length of paragraph not after a header.
\setlength{\parindent}{0cm}

% Remove page numbering in the beginning
\pagenumbering{gobble}

\begin{document}
    %% Change `ku-farve` to `nat-farve` to use SCIENCE's old colors or
    %% `natbio-farve` to use SCIENCE's new colors and logo.
    \AddToShipoutPicture*{\put(0,0){\includegraphics*[viewport=0 0 700 600]
        {include/ku-farve}}}
    \AddToShipoutPicture*{\put(0,602){\includegraphics*[viewport=0 600 700 1600]
        {include/ku-farve}}}

    %% Change `ku-en` to `nat-en` to use the `Faculty of Science` header
    \AddToShipoutPicture*{\put(0,0){\includegraphics*{include/ku-en}}}

    \clearpage
    \maketitle
    \thispagestyle{empty}

    \newpage

    %\tableofcontents

    %\newpage

    \pagenumbering{arabic} % Arabic page numbers (and reset to 1)

\section*{Task 1 - A priority queue}

\begin{lstlisting}[style=customc] 
typedef struct{
  void* thing;
  int priority;
}node;
\end{lstlisting}

\textit{Description}\\
A node containing elements needed for the priority queue. The heap will be build 
upon this type. thing contains a pointer to the element queued, and priority 
contains an integer. The higher value, the higher priority.\\
\textit{Assumptions}\\
Test\\
\textit{Running time}\\
Test



\begin{lstlisting}[style=customc] 
typedef struct{
  node* root;
  size_t size;
  size_t alloc_size;
}heap;
\end{lstlisting} 
\textit{Description}\\
The heap type.\\
\textit{Assumptions}\\
Test\\
\textit{Running time}\\
Test


 \begin{lstlisting}[style=customc] 
void heap_initialize(heap*);\end{lstlisting}

\textit{Description}\\
Setting up the node, memory assigning with malloc for one node element, and setting
the size variable of the heap type to 0.\\
\textit{Assumptions}\\
Test\\
\textit{Running time}\\
Test\\





\begin{lstlisting}[style=customc] 
void heap_clear(heap*); \end{lstlisting} 
\textit{Description}\\
Freeing up the assigned memory, and resetting the heap object by calling initialize again.\\
\textit{Assumptions}\\
Test\\
\textit{Running time}\\
Test



 \begin{lstlisting}[style=customc] 
size_t heap_size(heap*);\end{lstlisting} 

\textit{Description}\\
Returning the amount of elements in the heap. (stored under size in the heap type).\\
\textit{Assumptions}\\
Test\\
\textit{Running time}\\
Test





 \begin{lstlisting}[style=customc] 
void* heap_top(heap*);\end{lstlisting}

\textit{Description}\\
Since this function had a void* type instead of a node*, we assumed that only the void* thing was wanted for return, and not the whole node element. Since the heap structure is maximized, the return will always be the first element of the root array.\\
\textit{Assumptions}\\
Test\\
\textit{Running time}\\
Test

 


 \begin{lstlisting}[style=customc] 
void heap_insert(heap*,void*,int); \end{lstlisting} 

\textit{Description}\\
Inserts an element into the heap. The function arguments is an void pointer for the thing in the node type, and an intenger for the priority.\\ 
\textit{Assumptions}\\
Test\\
\textit{Running time}\\
Test


 \begin{lstlisting}[style=customc] 
void* heap_pop(heap*);\end{lstlisting}

\textit{Description}\\
Returns the first element in the heap, and max heapifys them.\\ 
\textit{Assumptions}\\
Test\\
\textit{Running time}\\
Test

  
\end{document}
