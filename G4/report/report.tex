\documentclass[11pt]{article}
\usepackage[a4paper, hmargin={2.8cm, 2.8cm}, vmargin={2.5cm, 2.5cm}]{geometry}
\usepackage{eso-pic} % \AddToShipoutPicture
\usepackage{graphicx} % \includegraphics
\usepackage{fancyhdr, amsmath, amssymb, comment, caption, placeins, subfigure,
    fixltx2e, changepage, listings, courier, soul, hyperref, geometry,
enumerate, listings, enumitem}
\usepackage[T1]{fontenc}
\usepackage[utf8]{inputenc}
\usepackage[english]{babel}


\author{\Large{Magnus N\o rskov Stavngaard} \\
    \texttt{magnus@stavngaard.dk}
    \\\\
    \Large{Mark Jan Jacobi} \\
    \texttt{mark@jacobi.pm}
    \\\\
    \Large{Mads Ynddal} \\
    \texttt{mynddal@me.com}
}

\lstdefinestyle{customc} {
    belowcaptionskip=1\baselineskip,
    breaklines=true,
    xleftmargin=\parindent,
    language=C,
    showstringspaces=false,
    keywordstyle=\bfseries\color{green!40!black},
    morekeywords={size_t,node,heap,pthread_mutex_t,node_t,queue_t,queue,semaphore_t,semstatus,spinlock_t,
    usr_sem_t_wrapper,usr_sem_t,uint32_t},
    commentstyle=\itshape\color{black},
    identifierstyle=\color{blue},
    stringstyle=\color{purple},
}

\lstset{
    basicstyle=\footnotesize,
    language=C,
numbers=none}
\lstset{literate=
    {á}{{\'a}}1 {é}{{\'e}}1 {í}{{\'i}}1 {ó}{{\'o}}1 {ú}{{\'u}}1
    {Á}{{\'A}}1 {É}{{\'E}}1 {Í}{{\'I}}1 {Ó}{{\'O}}1 {Ú}{{\'U}}1
    {à}{{\`a}}1 {è}{{\`e}}1 {ì}{{\`i}}1 {ò}{{\`o}}1 {ù}{{\`u}}1
    {À}{{\`A}}1 {È}{{\'E}}1 {Ì}{{\`I}}1 {Ò}{{\`O}}1 {Ù}{{\`U}}1
    {ä}{{\"a}}1 {ë}{{\"e}}1 {ï}{{\"i}}1 {ö}{{\"o}}1 {ü}{{\"u}}1
    {Ä}{{\"A}}1 {Ë}{{\"E}}1 {Ï}{{\"I}}1 {Ö}{{\"O}}1 {Ü}{{\"U}}1
    {â}{{\^a}}1 {ê}{{\^e}}1 {î}{{\^i}}1 {ô}{{\^o}}1 {û}{{\^u}}1
    {Â}{{\^A}}1 {Ê}{{\^E}}1 {Î}{{\^I}}1 {Ô}{{\^O}}1 {Û}{{\^U}}1
    {œ}{{\oe}}1 {Œ}{{\OE}}1 {æ}{{\ae}}1 {Æ}{{\AE}}1 {ß}{{\ss}}1
    {ç}{{\c c}}1 {Ç}{{\c C}}1 {ø}{{\o}}1 {å}{{\r a}}1 {Å}{{\r A}}1
    {€}{{\EUR}}1 {£}{{\pounds}}1
}

\title{
    \vspace{3cm}
    \Huge{OSM} \\
    \Large{G3}
}

\pagestyle{fancy}
\lhead{\small{Magnus S. Mark J. Mads Y.}}
\chead{\date{\today}}
\rhead{University of Copenhagen}
% \lfoot{}
% \cfoot{}
% \rfoot{}

% Change indent length of paragraph not after a header.
\setlength{\parindent}{0cm}

% Remove page numbering in the beginning
\pagenumbering{gobble}

\begin{document}
%% Change `ku-farve` to `nat-farve` to use SCIENCE's old colors or
%% `natbio-farve` to use SCIENCE's new colors and logo.
\AddToShipoutPicture*{\put(0,0){\includegraphics*[viewport=0 0 700 600]
{include/ku-farve}}}
\AddToShipoutPicture*{\put(0,602){\includegraphics*[viewport=0 600 700 1600]
{include/ku-farve}}}

%% Change `ku-en` to `nat-en` to use the `Faculty of Science` header
\AddToShipoutPicture*{\put(0,0){\includegraphics*{include/ku-en}}}

\clearpage
\maketitle
\thispagestyle{empty}

\newpage

%\tableofcontents

%\newpage

\pagenumbering{arabic} % Arabic page numbers (and reset to 1)

\section*{Task 1}
On the MIPS architecture, the TLB is handled in software. This means that the CPU calls interrupts, when a virtual memory address is tried to be accessed, while it is not in memory. When the interrupts are called, it is up to the operating system to handle and solve them.
Our implementation has some issues, but the idea is as follows: 
\begin{enumerate}
    \item Iterate through the page table, to find the Virtual Page Number (VPN) the instruction used.
    \item Check if the valid bit is set for the found page, if so, continue, otherwise go to 6.
    \item ???
    \item Overwrite a random page to solve the TLB interrupt
    \item ???
    \item Check if it is a kernel process. If so, do a kernel panic. If not, terminate the process.
\end{enumerate}

Our implementation fails, because it looks for an address that is off by a few bytes. A small test showed, that the difference between the correct address and address asked for is: (size of allocation)/2048 rounded up to nearest integer. A guess to why this error occurs, is because the VPN is exactly 2048 bytes and it might be allocated twice by mistake.

\section*{Task 2}
    We haven't made a working task 2, we used all the time we had trying to get
    task 1 to work so their wasn't any time for task 2.  We will here try to
    explain what task two should have done had it been implemented. \\

    In task two we were asked to create a function to help the implementation of
    malloc to allocate space on the heap.  The function to implement is called
    \texttt{syscall\_memlimit} and has the prototype,

    \begin{lstlisting}[style=customc]
void * syscall_memlimit(void *new_end);
    \end{lstlisting}

    The function should allocate more space for userland processes when they
    need it.  The function should try to move a heap pointer to the new point
    \texttt{heap\_end}, if this is possible the function should return a pointer
    to the new space allocated, if not the function should return \texttt{NULL}.
    If the function is given NULL as a parameter it should return the current
    end of the heap.  Normally a user process can both allocate and free memory
    from the heap, in Buenos freeing space on the heap is not implemented and
    therefore the heap can always only grow and never shrink.  This means that
    if a user attempts to decrease the heapsize the function should return an
    error, either a \texttt{NULL} pointer or a \texttt{KERNEL PANIC}.  \\

    The userland library \texttt{tests/lib.c} already contains an implementation
    of \texttt{malloc}, although this \texttt{malloc} implementation can only
    allocate 256 bytes.  If we had implemented \texttt{syscall\_memlimit} we
    could use this function in malloc when malloc runs out of memory to obtain
    more space from the system that can be used by the user.

\end{document}
