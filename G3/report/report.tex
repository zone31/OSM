\documentclass[11pt]{article}
\usepackage[a4paper, hmargin={2.8cm, 2.8cm}, vmargin={2.5cm, 2.5cm}]{geometry}
\usepackage{eso-pic} % \AddToShipoutPicture
\usepackage{graphicx} % \includegraphics
\usepackage{fancyhdr, amsmath, amssymb, comment, caption, placeins, subfigure,
    fixltx2e, changepage, listings, courier, soul, hyperref, geometry,
enumerate, listings, enumitem}
\usepackage[T1]{fontenc}
\usepackage[utf8]{inputenc}
\usepackage[english]{babel}


\author{\Large{Magnus N\o rskov Stavngaard} \\
    \texttt{magnus@stavngaard.dk}
    \\\\
    \Large{Mark Jan Jacobi} \\
    \texttt{mark@jacobi.pm}
    \\\\
    \Large{Mads Ynddal} \\
    \texttt{mynddal@me.com}
}

\lstdefinestyle{customc} {
    belowcaptionskip=1\baselineskip,
    breaklines=true,
    xleftmargin=\parindent,
    language=C,
    showstringspaces=false,
    keywordstyle=\bfseries\color{green!40!black},
    morekeywords={size_t,node,heap},
    commentstyle=\itshape\color{purple!40!black},
    identifierstyle=\color{blue},
    stringstyle=\color{orange},
}

\lstset{
    basicstyle=\footnotesize,
    language=C,
numbers=none}
\lstset{literate=
    {á}{{\'a}}1 {é}{{\'e}}1 {í}{{\'i}}1 {ó}{{\'o}}1 {ú}{{\'u}}1
    {Á}{{\'A}}1 {É}{{\'E}}1 {Í}{{\'I}}1 {Ó}{{\'O}}1 {Ú}{{\'U}}1
    {à}{{\`a}}1 {è}{{\`e}}1 {ì}{{\`i}}1 {ò}{{\`o}}1 {ù}{{\`u}}1
    {À}{{\`A}}1 {È}{{\'E}}1 {Ì}{{\`I}}1 {Ò}{{\`O}}1 {Ù}{{\`U}}1
    {ä}{{\"a}}1 {ë}{{\"e}}1 {ï}{{\"i}}1 {ö}{{\"o}}1 {ü}{{\"u}}1
    {Ä}{{\"A}}1 {Ë}{{\"E}}1 {Ï}{{\"I}}1 {Ö}{{\"O}}1 {Ü}{{\"U}}1
    {â}{{\^a}}1 {ê}{{\^e}}1 {î}{{\^i}}1 {ô}{{\^o}}1 {û}{{\^u}}1
    {Â}{{\^A}}1 {Ê}{{\^E}}1 {Î}{{\^I}}1 {Ô}{{\^O}}1 {Û}{{\^U}}1
    {œ}{{\oe}}1 {Œ}{{\OE}}1 {æ}{{\ae}}1 {Æ}{{\AE}}1 {ß}{{\ss}}1
    {ç}{{\c c}}1 {Ç}{{\c C}}1 {ø}{{\o}}1 {å}{{\r a}}1 {Å}{{\r A}}1
    {€}{{\EUR}}1 {£}{{\pounds}}1
}

\title{
    \vspace{3cm}
    \Huge{OSM} \\
    \Large{G3}
}

\pagestyle{fancy}
\lhead{\small{Magnus S. Mark J. Mads Y.}}
\chead{\date{\today}}
\rhead{University of Copenhagen}
% \lfoot{}
% \cfoot{}
% \rfoot{}

% Change indent length of paragraph not after a header.
\setlength{\parindent}{0cm}

% Remove page numbering in the beginning
\pagenumbering{gobble}

\begin{document}
%% Change `ku-farve` to `nat-farve` to use SCIENCE's old colors or
%% `natbio-farve` to use SCIENCE's new colors and logo.
\AddToShipoutPicture*{\put(0,0){\includegraphics*[viewport=0 0 700 600]
{include/ku-farve}}}
\AddToShipoutPicture*{\put(0,602){\includegraphics*[viewport=0 600 700 1600]
{include/ku-farve}}}

%% Change `ku-en` to `nat-en` to use the `Faculty of Science` header
\AddToShipoutPicture*{\put(0,0){\includegraphics*{include/ku-en}}}

\clearpage
\maketitle
\thispagestyle{empty}

\newpage

%\tableofcontents

%\newpage

\pagenumbering{arabic} % Arabic page numbers (and reset to 1)

\section{A thread-safe unbounded queue}



\subsection{Function Descriptions}


\begin{lstlisting}[style=customc]
typedef struct node {
    void *item;
    struct node *next;
} node_t;
\end{lstlisting}
\textit{Description} \\
This type includes a void pointer to an item and an pointer to the next element.
 This creates an linked list, where every typedef refers to its next element.
 this structure is only for a linked list, not for a queue.



\begin{lstlisting}[style=customc]
typedef struct queue {
    pthread_mutex_t queue_put_lock;
    pthread_mutex_t queue_get_lock;
    node_t *head;
    node_t *tail;
} queue_t;
\end{lstlisting}
\textit{Description} \\
The queue type contains a pointer to the head and tail of the node\_t linked list. Furthermore, it also
contains two pthread\_mutex\_t locks. These are used to lock the queue when adding or removing elements
from the linked list. The reason double pointers are used, instead of only the last element of the linked list, is because the function for returning the first element would have to traverse the full linked list.
When using 2 pointers, we do not need to traverse the list, and we don't need both top and bottom element to return the top.




\begin{lstlisting}[style=customc]
void *queue_get(queue_t *q);
\end{lstlisting}
\textit{Description} \\
Returns the first element of the queue, then sets the queue head element to the next in the list.
Also frees the memory.


\begin{lstlisting}[style=customc]
void queue_put(queue_t *q, void *item);
\end{lstlisting}
\textit{Description} \\
Adds an item to the linked list. Sets the tail pointer to this element, and sets the last node in the linked list to point at the new element as its next.



\begin{lstlisting}[style=customc]
void queue_init(queue_t *q);
\end{lstlisting}
\textit{Description} \\
Creates an empty queue element, with one node element inside. This is done so that getting an element from an empty queue can be handled easier. This way ,we know that a top node must have an next element. if not (NULL) we can return null instead of the init element.

\subsection{Thread locks}
The put and get functions use a mutex thread lock. This is done so multiple threads can't have access to the same element in the linked list.
\subsection{The test}
The test First generates a thousand threads. These threads will put and get the elements instant

\section{Userland semaphores for Buenos}
    Our implementation of semaphores in the buenos operating system is mainly
    located in the files \texttt{proc/semaphore.c} and
    \texttt{proc/semaphore.h}.  Our implementation is made around a
    datastructure representing a semaphore located in \texttt{proc/semaphore.h},

    \begin{lstlisting}[style=customc]
typedef struct {
    semaphore_t *kernel_semaphore; /* A pointer to the kernel impl. */
    semstatus status; /* The status of the semaphore. */
    char name[SEM_NAME_MAX_LENGTH]; /* The name of the semaphore. */
    int ident; /* The position in the array of semaphores. */
    int initial_value;
    spinlock_t slock;
} usr_sem_t_wrapper;
    \end{lstlisting}

    The datastructure contains,

    \begin{itemize}
        \item a pointer to a kernel semaphore, this is where the actual
            implementation of the semaphore is.
        \item a semstatus status, where we keep the current status of the
            semaphore.  The semaphore can be either \texttt{SEM\_ALLOCED} or
            \texttt{SEM\_NONALLOCED} depending on it being in use or not.
        \item a char array of size \texttt{SEM\_NAME\_MAX\_LENGTH} containing the
            name of the semaphore.  The length of a given name is limited to
            the size of this array.  The name is used to find a specific
            semaphore.
        \item an integer ident.  This integer is used to obscure where a
            semaphore is actually located so a user cannot tamper with it. (this
            will be described in more detail later.)
        \item an integer initial\_value.  This is used to compare the current
            value of a semaphore with the initial value to find out if it is
            alright to remove a semaphore or if it is in use.
        \item a spinlock slock, used to make sure that no process is running
            concurrently with a semaphore being deleted as this can lead to
            errors.
    \end{itemize}

    This structure is never returned explicitly to the user, but is only used
    internally in the kernel.  The structure returned to the used is defined in
    \texttt{tests/lib.h} as,

    \begin{lstlisting}[style=customc]
typedef void usr_sem_t;
    \end{lstlisting}

    this means that when we in a function returns a pointer to a
    \texttt{usr\_sem\_t}, we are actually returning a \texttt{void} pointer,
    which can be anything.  In our implementation the void pointer actually
    points to the integer ident from the semaphore structure, this integer is
    then an index in an array of semaphores which we holds.  A user therefore
    only has direct access to a other semaphores if he/she knows where the array
    starts.

    \subsection{Operations on Semaphores}
    The operations that can be performed on the semaphores are,

    \begin{lstlisting}[style=customc]
usr_sem_t * syscall_sem_open(char const *name, int value);
int syscall_sem_p(usr_sem_t *handle);
int syscall_sem_v(usr_sem_t *handle);
int syscall_sem_destroy(usr_sem_t *handle);
void sem_user_init(void);
    \end{lstlisting}

    where the functions \texttt{syscall\_sem\_open}, \texttt{syscall\_sem\_p},
    \texttt{syscall\_sem\_v} and \texttt{syscall\_sem\_destroy} is available for
    users to call and \texttt{sem\_user\_init} only can be reached by the
    kernel.  We have implemented the system calls in the file
    \texttt{tests/lib.c} with system call numbers located in
    \texttt{proc/syscall.h}.  The implementation of \texttt{syscall\_sem\_open}
    in the \texttt{tests/lib.c} is shown here, the rest of the syscalls is very
    similar. \\

    The syscall is implemented by,

    \begin{lstlisting}[style=customc]
usr_sem_t* syscall_sem_open(const char* name, int value)
{
    return (usr_sem_t*) _syscall(SYSCALL_SEM_OPEN, (uint32_t) name,
        (uint32_t) value, 0);
}
    \end{lstlisting}

    We call the \texttt{\_syscall} function with the code
    \texttt{SYSCALL\_SEM\_OPEN}, which makes the switch in
    \texttt{proc/syscall.c} handle a \texttt{syscall\_sem\_open}. \\

    We will now describe the actual implementation of the function operating on
    a semaphore.

    \subsection{syscall\_sem\_open}
    To create a new semaphore or get an already existing semaphore
    \texttt{syscall\_sem\_open} is called.  The function takes two parameters,
    a char const pointer and an integer.  The char const pointer is used to
    identify a semaphore and is saved in the structure representing the
    semaphore.  The value is used for two things, if the value is under 0 it is
    discarded and the function searches for a semaphore of the given name, if
    value is greater or equal to 0 a new semaphore is created with that value.
    The code is,

    \begin{lstlisting}[style=customc]
usr_sem_t * syscall_sem_open(char const *name, int value)
{
    if (value >= 0) {
        return &(create_semaphore(name, value)->ident);
    } else {
        return &(get_semaphore(name)->ident);
    }
}
    \end{lstlisting}

    To create a new semaphore we simply loop through our array of semaphores and
    find a semaphore where the status is equal to \texttt{SEM\_NONALLOCED}.  We
    then change the status and set the name as the given name and initialize the
    kernel semaphore with the value given.  If any error occurs we return a
    \texttt{NULL} pointer. \\

    To find a already existing semaphore we search through the array of
    semaphores for a similar name as the one given.  If it is found, we return
    a pointer to that semaphore, if not we return a \texttt{NULL} pointer.

    \subsection{syscall\_sem\_init}
    This function can only be called from the kernel, the function has to be
    called before any userland semaphores are used.  The function is called from
    \texttt{init/main.c}.  The function initializes all semaphores to be
    unalloced, their names to be a zero byte and their ident to be their
    position in the array.  The function is defined as,

    \begin{lstlisting}[style=customc]
void sem_user_init(void)
{
    int i;

    for (i = 0; i < SEM_MAX_SEMAPHORES; i++){
        semaphores[i].status = SEM_NONALLOCED;
        semaphores[i].name[0] = '\0';
        semaphores[i].ident = i;
    }
}
    \end{lstlisting}

\subsection{syscall\_sem\_p}
To \emph{procure} (reserve) a resource, we implemented the \emph{syscall\_sem\_p(handle)} system call. It takes a semaphore handle as argument, that was previously returned by the \emph{syscall\_sem\_open(name, value)} call. The handle is used only internally by the userland semaphores. The kernel semaphore is hidden inside the \emph{struct usr\_sem\_t\_wrapper}.

After checking the argument for runtime errors, the kernel semaphore is retrieved, and the procure function from the kernel is called.

% int syscall_sem_p(usr_sem_t* handle)
% {
%     int index;
%     usr_sem_t_wrapper *handle_wrapper;

%     /* Return error if handle is NULL. */
%     KERNEL_ASSERT(handle != NULL);

%     /* Get the index in the array. */
%     index = *((int *) handle);

%     /* Get the actual handle and procure a resource. */
%     handle_wrapper = &(semaphores[index]);

%     if (handle_wrapper->status == SEM_NONALLOCED) {
%         return SEM_DOES_NOT_EXIST;
%     }

%     semaphore_P(handle_wrapper->kernel_semaphore);

%     return 0;
% }

\subsection{syscall\_sem\_v}
To vacate (release) a resource, we implemented the \emph{syscall\_sem\_v} system call in nearly the same way as \emph{syscall\_sem\_p}. The function takes a handle as argument and retrieves the kernel semaphore. The kernel function \emph{semaphore\_V} is called with the retrieved kernel semaphore as argument.

% int syscall_sem_v(usr_sem_t* handle)
% {
%     int index;
%     usr_sem_t_wrapper *handle_wrapper;

%     /* Return error if handle is NULL. */
%     KERNEL_ASSERT(handle != NULL);

%     /* Get the index in the array. */
%     index = *((int *) handle);

%     /* Get the actual handle and vacate the resource. */
%     handle_wrapper = &(semaphores[index]);

%     if (handle_wrapper->status == SEM_NONALLOCED) {
%         return SEM_DOES_NOT_EXIST;
%     }

%     semaphore_V(handle_wrapper->kernel_semaphore);

%     return 0;
% }

\subsection{syscall\_sem\_destroy}
When a semaphore is no longer needed. It can safely be discarded with the \emph{syscall\_sem\_destroy(handle)} call. The function will prevent simple runtime errors and afterwards check if the semaphore is still in use ie. a function is waiting for a vacate call. To ensure, that the function is not interrupted, all interrupts are disabled, and the function acquires the handle's spinlock. If nothing is using the semaphore, the semaphore is set to \emph{SEM\_NONALLOCED} and is now able to get reallocated in a \emph{syscall\_sem\_open} call.

% /* Removes a semaphore with a given handle, the handle is dereferenced to be an
%  * integer corresponding to an index in the array of semaphores.  This index is
%  * used to find a semaphore and remove it. */
% int syscall_sem_destroy(usr_sem_t* handle)
% {
%     interrupt_status_t intr_status;

%     usr_sem_t_wrapper *handle_wrapper = &semaphores[*((int *)handle)];

%     /* Semaphore doesn't exist. */
%     if (handle_wrapper->status == SEM_NONALLOCED) {
%         return SEM_DOES_NOT_EXIST;
%     }

%     /* Lock when testing if the semaphore is ready to be removed. */
%     intr_status = _interrupt_disable();
%     spinlock_acquire(&handle_wrapper->slock);

%     /* Test if the semaphore is done being used. */
%     if (handle_wrapper->initial_value !=
%         handle_wrapper->kernel_semaphore->value) {

%         /* Release the lock. */
%         spinlock_release(&handle_wrapper->slock);
%         _interrupt_set_state(intr_status);

%         return SEM_IN_USE;
%     }

%     handle_wrapper->status = SEM_NONALLOCED;

%     /* Destroy kernel semaphore. */
%     semaphore_destroy(handle_wrapper->kernel_semaphore);

%     /* Release the lock. */
%     spinlock_release(&handle_wrapper->slock);
%     _interrupt_set_state(intr_status);

%     return SUCCESS;
% }

\end{document}
