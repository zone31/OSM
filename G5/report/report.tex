\documentclass[11pt]{article}
\usepackage[a4paper, hmargin={2.8cm, 2.8cm}, vmargin={2.5cm, 2.5cm}]{geometry}
\usepackage{eso-pic} % \AddToShipoutPicture
\usepackage{graphicx} % \includegraphics
\usepackage{fancyhdr, amsmath, amssymb, comment, caption, placeins, subfigure,
    fixltx2e, changepage, listings, courier, soul, hyperref, geometry,
enumerate, listings, enumitem}
\usepackage[T1]{fontenc}
\usepackage[utf8]{inputenc}
\usepackage[english]{babel}


\author{\Large{Magnus N\o rskov Stavngaard} \\
    \texttt{magnus@stavngaard.dk}
    \\\\
    \Large{Mark Jan Jacobi} \\
    \texttt{mark@jacobi.pm}
    \\\\
    \Large{Mads Ynddal} \\
    \texttt{mynddal@me.com}
}

\lstdefinestyle{customc} {
    belowcaptionskip=1\baselineskip,
    breaklines=true,
    xleftmargin=\parindent,
    language=C,
    showstringspaces=false,
    keywordstyle=\bfseries\color{green!40!black},
    morekeywords={size_t,node,heap,pthread_mutex_t,node_t,queue_t,queue,semaphore_t,semstatus,spinlock_t,
    usr_sem_t_wrapper,usr_sem_t,uint32_t},
    commentstyle=\itshape\color{black},
    identifierstyle=\color{blue},
    stringstyle=\color{purple},
}

\lstset{
    basicstyle=\footnotesize,
    language=C,
numbers=none}
\lstset{literate=
    {á}{{\'a}}1 {é}{{\'e}}1 {í}{{\'i}}1 {ó}{{\'o}}1 {ú}{{\'u}}1
    {Á}{{\'A}}1 {É}{{\'E}}1 {Í}{{\'I}}1 {Ó}{{\'O}}1 {Ú}{{\'U}}1
    {à}{{\`a}}1 {è}{{\`e}}1 {ì}{{\`i}}1 {ò}{{\`o}}1 {ù}{{\`u}}1
    {À}{{\`A}}1 {È}{{\'E}}1 {Ì}{{\`I}}1 {Ò}{{\`O}}1 {Ù}{{\`U}}1
    {ä}{{\"a}}1 {ë}{{\"e}}1 {ï}{{\"i}}1 {ö}{{\"o}}1 {ü}{{\"u}}1
    {Ä}{{\"A}}1 {Ë}{{\"E}}1 {Ï}{{\"I}}1 {Ö}{{\"O}}1 {Ü}{{\"U}}1
    {â}{{\^a}}1 {ê}{{\^e}}1 {î}{{\^i}}1 {ô}{{\^o}}1 {û}{{\^u}}1
    {Â}{{\^A}}1 {Ê}{{\^E}}1 {Î}{{\^I}}1 {Ô}{{\^O}}1 {Û}{{\^U}}1
    {œ}{{\oe}}1 {Œ}{{\OE}}1 {æ}{{\ae}}1 {Æ}{{\AE}}1 {ß}{{\ss}}1
    {ç}{{\c c}}1 {Ç}{{\c C}}1 {ø}{{\o}}1 {å}{{\r a}}1 {Å}{{\r A}}1
    {€}{{\EUR}}1 {£}{{\pounds}}1
}

\title{
    \vspace{3cm}
    \Huge{OSM} \\
    \Large{G5}
}

\pagestyle{fancy}
\lhead{\small{Magnus S. Mark J. Mads Y.}}
\chead{\date{\today}}
\rhead{University of Copenhagen}
% \lfoot{}
% \cfoot{}
% \rfoot{}

% Change indent length of paragraph not after a header.
\setlength{\parindent}{0cm}

% Remove page numbering in the beginning
\pagenumbering{gobble}

\begin{document}
%% Change `ku-farve` to `nat-farve` to use SCIENCE's old colors or
%% `natbio-farve` to use SCIENCE's new colors and logo.
\AddToShipoutPicture*{\put(0,0){\includegraphics*[viewport=0 0 700 600]
{include/ku-farve}}}
\AddToShipoutPicture*{\put(0,602){\includegraphics*[viewport=0 600 700 1600]
{include/ku-farve}}}

%% Change `ku-en` to `nat-en` to use the `Faculty of Science` header
\AddToShipoutPicture*{\put(0,0){\includegraphics*{include/ku-en}}}

\clearpage
\maketitle
\thispagestyle{empty}

\newpage

\tableofcontents

\newpage

\pagenumbering{arabic} % Arabic page numbers (and reset to 1)

\section{Task 1}
In this task we were asked to implement a number of system calls in relation to
the filesystem.  The calls are supposed to give user processes permission to
create, open, read, write and delete files.  To do that we mainly create wrapper
functions to an already existing simple filesystem. \\

The system calls we have implemented are explained under this.


\subsection{syscall\_open}
The function is defined as such:
    \begin{lstlisting}[style=customc]
int syscall_open(const char* pathname);
    \end{lstlisting}

This function returns an file handle integer. This is later used to identify 
the file for other file modifying functions, such as $read,write,tell,seek,close$.
The pathname needs to have the volume name in the beginning, and then the file path.
This is defined as such : $[disk]file$.\\

In the implementation, the userspace program calls $syscall\_open()$ in the lib.c file, which wraps the call
into the $syscall.c$ via mips assembly. This then calls the $vfs\_open$. In the vfs\_open file, we keep a list of files.
this list is not dynamic, and can contain $512$ files. \\

The first 3 files are reserved for the system signals, $IN$, $OUT$ and $ERROR$. These are mapped by numbers, where $IN=0$,$OUT=1$,$ERROR=2$. Since the file handle integer return is directly mapped to the list of files, we then occupy the first 3 at initiation. \\

This also makes the file systemcalls able to modify the standard input, output and error, by setting the filehandle accordingly.




\subsection{syscall\_close}
Syscall close is defined as:

    \begin{lstlisting}[style=customc]
int syscall_close(int filehandle);
    \end{lstlisting}

The function takes a filehandle to an open file and closes it.  To do that we
use the function \texttt{vfs\_close} defined in the file \texttt{fs/vfs.c}.  This
function also takes a filehandle to an open file and first test if the
file is really open.  If the file is open the file is closed, otherwise an error
is returned. \\

We have made the function available to the user by supplying an interface
through the system call \texttt{SYSCALL\_CLOSE}, the system call is implemented
in \texttt{proc/syscall.c} as,

\begin{lstlisting}[style=customc]
case SYSCALL_CLOSE:
    V0 = vfs_close((int) A1);
    break;
\end{lstlisting}

And this system call is made available through a implementation in
\texttt{tests/lib.c} which looks like,

\begin{lstlisting}[style=customc]
int syscall_close(int filehandle)
{
    return (int)_syscall(SYSCALL_CLOSE, (uint32_t)filehandle, 0, 0);
}
\end{lstlisting}

Therefore a user can close an open file by making a call to
\texttt{syscall\_close}.

\subsection{syscall\_create}
Syscall create is defined as:
    \begin{lstlisting}[style=customc]
int syscall_create(const char* pathname, int size);
    \end{lstlisting}
syscall\_create registers a file on the disk with the specified file size. The file can then be opened, written to and read from by the syscalls described in this section.
The syscall is implemented by calling the vfs\_create function defined by the virtual filesystem in the kernel.


\subsection{syscall\_remove}
Syscall remove is defined as:
    \begin{lstlisting}[style=customc]
int syscall_remove(const char* pathname);
    \end{lstlisting}
Syscall remove is used to remove a file, the syscall is implemented by calling
the function, \texttt{vfs\_remove}, from the file \texttt{fs/vfs.c}.  The
function removes the file from the filesystem by calling a filesystem dependent
function and returns 0 if the removal was successful and an error code
otherwise. \\

The function is made available to the user through a system call,
\texttt{SYSCALL\_REMOVE}, implemented as,

\begin{lstlisting}[style=customc]
case SYSCALL_REMOVE:
    V0 = vfs_remove((char const *) A1);
    break;
\end{lstlisting}

and this system call is made available through a wrapper in \texttt{tests/lib.c}
implemented as,

\begin{lstlisting}[style=customc]
int syscall_delete(const char *filename)
{
    return (int)_syscall(SYSCALL_REMOVE, (uint32_t)filename, 0, 0);
}
\end{lstlisting}

So to call \texttt{syscall\_remove} one has to call the function
\texttt{syscall\_delete}.



\subsection{syscall\_seek}
Syscall seek is defined as:
    \begin{lstlisting}[style=customc]
int syscall_seek(int filehandle, int offset);
    \end{lstlisting}

syscall\_seek changes the offset to read and write from for a filehandle.
The syscall is implemented, by finding the filehandle and acquire a semaphore lock. Then the seek\_position variable is changed to the absolute postion defined in the parameter to the function. The semphore is released before the function returns.


\subsection{syscall\_tell}
Syscall tell is defined as:
    \begin{lstlisting}[style=customc]
int syscall_tell(int filehandle)
    \end{lstlisting}
The function tell the current position in the file relative the start of the
file.  The position returned is where the next byte is to be read or written,
we have implemented \texttt{tell} in the file \texttt{fs/vfs.c}.  The
implementation is,

\begin{lstlisting}[style=customc]
int vfs_tell(openfile_t file)
{
    openfile_entry_t *openfile;

    if (vfs_start_op() != VFS_OK)
        return VFS_UNUSABLE;

    semaphore_P(openfile_table.sem);

    openfile = vfs_verify_open(file);
    if (openfile == NULL) {
        semaphore_V(openfile_table.sem);
        return VFS_NOT_OPEN;
    }

    int seek = openfile->seek_position;
    semaphore_V(openfile_table.sem);

    vfs_end_op();
    return seek;
}
\end{lstlisting}

The function test if the file given is open, if it is open it finds the seek
position in the openfile type and return this position.  The function is
wrapped in a semaphore to make sure no one is messing with the file while we
read the position.  A user has access to the function through a syscall
interface.  We have implemented wrappers in \texttt{proc/syscall.c} and
\texttt{tests/lib.c}.

\subsection{syscall\_read}
Syscall read is defined as:
    \begin{lstlisting}[style=customc]
int syscall_read(int filehandle, void* buffer, int length);
    \end{lstlisting}

syscall\_read reads a series of bytes from a filehandle into a buffer. To avoid reading more than expected (causing an unhandled buffer overflow), a length is defnied as parameter.
The read syscall is handled by sorting out, what kind of read is needed. The function io\_read determines if the filehandle is STDIN or a file. In case of TTY, the tty\_read\_stdin call is used, otherwise vfs\_write is called.

\subsection{syscall\_write}
Syscall write is defined as:
    \begin{lstlisting}[style=customc]
int syscall_write(int filehandle, void const* buffer, int length)
    \end{lstlisting}
syscall\_write writes a series of bytes to a filehandle from a buffer. To avoid writing longer than expected (causing an unhandled buffer overflow), a length is defined as parameter.
The write syscall is handled by sorting out, what kind of write is needed. The function io\_write determines if the filehandle is STDOUT, STDERR or a file. In case of TTY, the tty\_write\_stdout and tty\_write\_stderr calls are used, otherwise vfs\_write is called.


\section{Task 2}
In this task we were asked to implement a better filesystem than the simple one
that was shipped with Buenos.  When we create a file in Buenos we have to know
its size beforehand, the filesystem can't itself expand a file naturally as it
is written.  We were asked to implement a function that allocates more space on
the filesystem if a write goes over the bound of the maximum filelength.

\end{document}
