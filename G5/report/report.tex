\documentclass[11pt]{article}
\usepackage[a4paper, hmargin={2.8cm, 2.8cm}, vmargin={2.5cm, 2.5cm}]{geometry}
\usepackage{eso-pic} % \AddToShipoutPicture
\usepackage{graphicx} % \includegraphics
\usepackage{fancyhdr, amsmath, amssymb, comment, caption, placeins, subfigure,
    fixltx2e, changepage, listings, courier, soul, hyperref, geometry,
enumerate, listings, enumitem}
\usepackage[T1]{fontenc}
\usepackage[utf8]{inputenc}
\usepackage[english]{babel}


\author{\Large{Magnus N\o rskov Stavngaard} \\
    \texttt{magnus@stavngaard.dk}
    \\\\
    \Large{Mark Jan Jacobi} \\
    \texttt{mark@jacobi.pm}
    \\\\
    \Large{Mads Ynddal} \\
    \texttt{mynddal@me.com}
}

\lstdefinestyle{customc} {
    belowcaptionskip=1\baselineskip,
    breaklines=true,
    xleftmargin=\parindent,
    language=C,
    showstringspaces=false,
    keywordstyle=\bfseries\color{green!40!black},
    morekeywords={size_t,node,heap,pthread_mutex_t,node_t,queue_t,queue,semaphore_t,semstatus,spinlock_t,
    usr_sem_t_wrapper,usr_sem_t,uint32_t},
    commentstyle=\itshape\color{black},
    identifierstyle=\color{blue},
    stringstyle=\color{purple},
}

\lstset{
    basicstyle=\footnotesize,
    language=C,
numbers=none}
\lstset{literate=
    {á}{{\'a}}1 {é}{{\'e}}1 {í}{{\'i}}1 {ó}{{\'o}}1 {ú}{{\'u}}1
    {Á}{{\'A}}1 {É}{{\'E}}1 {Í}{{\'I}}1 {Ó}{{\'O}}1 {Ú}{{\'U}}1
    {à}{{\`a}}1 {è}{{\`e}}1 {ì}{{\`i}}1 {ò}{{\`o}}1 {ù}{{\`u}}1
    {À}{{\`A}}1 {È}{{\'E}}1 {Ì}{{\`I}}1 {Ò}{{\`O}}1 {Ù}{{\`U}}1
    {ä}{{\"a}}1 {ë}{{\"e}}1 {ï}{{\"i}}1 {ö}{{\"o}}1 {ü}{{\"u}}1
    {Ä}{{\"A}}1 {Ë}{{\"E}}1 {Ï}{{\"I}}1 {Ö}{{\"O}}1 {Ü}{{\"U}}1
    {â}{{\^a}}1 {ê}{{\^e}}1 {î}{{\^i}}1 {ô}{{\^o}}1 {û}{{\^u}}1
    {Â}{{\^A}}1 {Ê}{{\^E}}1 {Î}{{\^I}}1 {Ô}{{\^O}}1 {Û}{{\^U}}1
    {œ}{{\oe}}1 {Œ}{{\OE}}1 {æ}{{\ae}}1 {Æ}{{\AE}}1 {ß}{{\ss}}1
    {ç}{{\c c}}1 {Ç}{{\c C}}1 {ø}{{\o}}1 {å}{{\r a}}1 {Å}{{\r A}}1
    {€}{{\EUR}}1 {£}{{\pounds}}1
}

\title{
    \vspace{3cm}
    \Huge{OSM} \\
    \Large{G3}
}

\pagestyle{fancy}
\lhead{\small{Magnus S. Mark J. Mads Y.}}
\chead{\date{\today}}
\rhead{University of Copenhagen}
% \lfoot{}
% \cfoot{}
% \rfoot{}

% Change indent length of paragraph not after a header.
\setlength{\parindent}{0cm}

% Remove page numbering in the beginning
\pagenumbering{gobble}

\begin{document}
%% Change `ku-farve` to `nat-farve` to use SCIENCE's old colors or
%% `natbio-farve` to use SCIENCE's new colors and logo.
\AddToShipoutPicture*{\put(0,0){\includegraphics*[viewport=0 0 700 600]
{include/ku-farve}}}
\AddToShipoutPicture*{\put(0,602){\includegraphics*[viewport=0 600 700 1600]
{include/ku-farve}}}

%% Change `ku-en` to `nat-en` to use the `Faculty of Science` header
\AddToShipoutPicture*{\put(0,0){\includegraphics*{include/ku-en}}}

\clearpage
\maketitle
\thispagestyle{empty}

\newpage

%\tableofcontents

%\newpage

\pagenumbering{arabic} % Arabic page numbers (and reset to 1)

\section{Task 1}
In this task we were asked to implement a number of system calls in relation to
the filesystem.  The calls are supposed to give user processes permission to
create, open, read, write and delete files.  To do that we mainly create wrapper
functions to an already existing simple filesystem. \\

The system calls we have implemented are explained under this.

\subsection{syscall\_close}
Syscall close is defined as,

    \begin{lstlisting}[style=customc]
int syscall_close(int filehandle);
    \end{lstlisting}

The function takes a filehandle to an open file and closes it.  To do that we
use the function \texttt{vfs\_close} defined in the file \texttt{fs/vfs.c}.  This
function also takes a filehandle to an open file and first test if the
file is really open.  If the file is open the file is closed, otherwise an error
is returned. \\

We have made the function available to the user by supplying an interface
through the system call \texttt{SYSCALL\_CLOSE}, the system call is implemented
in \texttt{proc/syscall.c} as,

\begin{lstlisting}[style=customc]
case SYSCALL_CLOSE:
    V0 = vfs_close((int) A1);
    break;
\end{lstlisting}

And this system call is made available through a implementation in
\texttt{tests/lib.c} which looks like,

\begin{lstlisting}[style=customc]
int syscall_close(int filehandle)
{
    return (int)_syscall(SYSCALL_CLOSE, (uint32_t)filehandle, 0, 0);
}
\end{lstlisting}

Therefore a user can close an open file by making a call to
\texttt{syscall\_close}.

\subsection{syscall\_remove}
Syscall remove is used to remove a file, the syscall is implemented by calling
the function, \texttt{vfs\_remove}, from the file \texttt{fs/vfs.c}.  The
function removes the file from the filesystem by calling a filesystem dependent
function and returns 0 if the removal was successful and an error code
otherwise. \\

The function is made available to the user through a system call,
\texttt{SYSCALL\_REMOVE}, implemented as,

\begin{lstlisting}[style=customc]
case SYSCALL_REMOVE:
    V0 = vfs_remove((char const *) A1);
    break;
\end{lstlisting}

and this system call is made available through a wrapper in \texttt{tests/lib.c}
implemented as,

\begin{lstlisting}[style=customc]
int syscall_delete(const char *filename)
{
    return (int)_syscall(SYSCALL_REMOVE, (uint32_t)filename, 0, 0);
}
\end{lstlisting}

So to call \texttt{syscall\_remove} one has to call the function
\texttt{syscall\_delete}.

\subsection{syscall\_tell}
The function tell the current position in the file relative the start of the
file.  The position returned is where the next byte is to be read or written,
we have implemented \texttt{tell} in the file \texttt{fs/vfs.c}.  The
implementation is,

\begin{lstlisting}[style=customc]
int vfs_tell(openfile_t file)
{
    openfile_entry_t *openfile;

    if (vfs_start_op() != VFS_OK)
        return VFS_UNUSABLE;

    semaphore_P(openfile_table.sem);

    openfile = vfs_verify_open(file);
    if (openfile == NULL) {
        semaphore_V(openfile_table.sem);
        return VFS_NOT_OPEN;
    }

    int seek = openfile->seek_position;
    semaphore_V(openfile_table.sem);

    vfs_end_op();
    return seek;
}
\end{lstlisting}

The function test if the file given is open, if it is open it finds the seek
position in the openfile type and return this position.  The function is
wrapped in a semaphore to make sure no one is messing with the file while we
read the position.  A user has access to the function through a syscall
interface.  We have implemented wrappers in \texttt{proc/syscall.c} and
\texttt{tests/lib.c}.


\end{document}
